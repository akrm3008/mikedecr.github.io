\documentclass[]{article}
\usepackage{lmodern}
\usepackage{amssymb,amsmath}
\usepackage{ifxetex,ifluatex}
\usepackage{fixltx2e} % provides \textsubscript
\ifnum 0\ifxetex 1\fi\ifluatex 1\fi=0 % if pdftex
  \usepackage[T1]{fontenc}
  \usepackage[utf8]{inputenc}
\else % if luatex or xelatex
  \ifxetex
    \usepackage{mathspec}
  \else
    \usepackage{fontspec}
  \fi
  \defaultfontfeatures{Ligatures=TeX,Scale=MatchLowercase}
\fi
% use upquote if available, for straight quotes in verbatim environments
\IfFileExists{upquote.sty}{\usepackage{upquote}}{}
% use microtype if available
\IfFileExists{microtype.sty}{%
\usepackage{microtype}
\UseMicrotypeSet[protrusion]{basicmath} % disable protrusion for tt fonts
}{}
\usepackage[margin=1in]{geometry}
\usepackage{hyperref}
\hypersetup{unicode=true,
            pdftitle={Packages \& Reproducibility: Install what you need, attach what you want},
            pdfauthor={Mike DeCrescenzo},
            pdfborder={0 0 0},
            breaklinks=true}
\urlstyle{same}  % don't use monospace font for urls
\usepackage{color}
\usepackage{fancyvrb}
\newcommand{\VerbBar}{|}
\newcommand{\VERB}{\Verb[commandchars=\\\{\}]}
\DefineVerbatimEnvironment{Highlighting}{Verbatim}{commandchars=\\\{\}}
% Add ',fontsize=\small' for more characters per line
\usepackage{framed}
\definecolor{shadecolor}{RGB}{248,248,248}
\newenvironment{Shaded}{\begin{snugshade}}{\end{snugshade}}
\newcommand{\AlertTok}[1]{\textcolor[rgb]{0.94,0.16,0.16}{#1}}
\newcommand{\AnnotationTok}[1]{\textcolor[rgb]{0.56,0.35,0.01}{\textbf{\textit{#1}}}}
\newcommand{\AttributeTok}[1]{\textcolor[rgb]{0.77,0.63,0.00}{#1}}
\newcommand{\BaseNTok}[1]{\textcolor[rgb]{0.00,0.00,0.81}{#1}}
\newcommand{\BuiltInTok}[1]{#1}
\newcommand{\CharTok}[1]{\textcolor[rgb]{0.31,0.60,0.02}{#1}}
\newcommand{\CommentTok}[1]{\textcolor[rgb]{0.56,0.35,0.01}{\textit{#1}}}
\newcommand{\CommentVarTok}[1]{\textcolor[rgb]{0.56,0.35,0.01}{\textbf{\textit{#1}}}}
\newcommand{\ConstantTok}[1]{\textcolor[rgb]{0.00,0.00,0.00}{#1}}
\newcommand{\ControlFlowTok}[1]{\textcolor[rgb]{0.13,0.29,0.53}{\textbf{#1}}}
\newcommand{\DataTypeTok}[1]{\textcolor[rgb]{0.13,0.29,0.53}{#1}}
\newcommand{\DecValTok}[1]{\textcolor[rgb]{0.00,0.00,0.81}{#1}}
\newcommand{\DocumentationTok}[1]{\textcolor[rgb]{0.56,0.35,0.01}{\textbf{\textit{#1}}}}
\newcommand{\ErrorTok}[1]{\textcolor[rgb]{0.64,0.00,0.00}{\textbf{#1}}}
\newcommand{\ExtensionTok}[1]{#1}
\newcommand{\FloatTok}[1]{\textcolor[rgb]{0.00,0.00,0.81}{#1}}
\newcommand{\FunctionTok}[1]{\textcolor[rgb]{0.00,0.00,0.00}{#1}}
\newcommand{\ImportTok}[1]{#1}
\newcommand{\InformationTok}[1]{\textcolor[rgb]{0.56,0.35,0.01}{\textbf{\textit{#1}}}}
\newcommand{\KeywordTok}[1]{\textcolor[rgb]{0.13,0.29,0.53}{\textbf{#1}}}
\newcommand{\NormalTok}[1]{#1}
\newcommand{\OperatorTok}[1]{\textcolor[rgb]{0.81,0.36,0.00}{\textbf{#1}}}
\newcommand{\OtherTok}[1]{\textcolor[rgb]{0.56,0.35,0.01}{#1}}
\newcommand{\PreprocessorTok}[1]{\textcolor[rgb]{0.56,0.35,0.01}{\textit{#1}}}
\newcommand{\RegionMarkerTok}[1]{#1}
\newcommand{\SpecialCharTok}[1]{\textcolor[rgb]{0.00,0.00,0.00}{#1}}
\newcommand{\SpecialStringTok}[1]{\textcolor[rgb]{0.31,0.60,0.02}{#1}}
\newcommand{\StringTok}[1]{\textcolor[rgb]{0.31,0.60,0.02}{#1}}
\newcommand{\VariableTok}[1]{\textcolor[rgb]{0.00,0.00,0.00}{#1}}
\newcommand{\VerbatimStringTok}[1]{\textcolor[rgb]{0.31,0.60,0.02}{#1}}
\newcommand{\WarningTok}[1]{\textcolor[rgb]{0.56,0.35,0.01}{\textbf{\textit{#1}}}}
\usepackage{longtable,booktabs}
\usepackage{graphicx,grffile}
\makeatletter
\def\maxwidth{\ifdim\Gin@nat@width>\linewidth\linewidth\else\Gin@nat@width\fi}
\def\maxheight{\ifdim\Gin@nat@height>\textheight\textheight\else\Gin@nat@height\fi}
\makeatother
% Scale images if necessary, so that they will not overflow the page
% margins by default, and it is still possible to overwrite the defaults
% using explicit options in \includegraphics[width, height, ...]{}
\setkeys{Gin}{width=\maxwidth,height=\maxheight,keepaspectratio}
\IfFileExists{parskip.sty}{%
\usepackage{parskip}
}{% else
\setlength{\parindent}{0pt}
\setlength{\parskip}{6pt plus 2pt minus 1pt}
}
\setlength{\emergencystretch}{3em}  % prevent overfull lines
\providecommand{\tightlist}{%
  \setlength{\itemsep}{0pt}\setlength{\parskip}{0pt}}
\setcounter{secnumdepth}{5}
% Redefines (sub)paragraphs to behave more like sections
\ifx\paragraph\undefined\else
\let\oldparagraph\paragraph
\renewcommand{\paragraph}[1]{\oldparagraph{#1}\mbox{}}
\fi
\ifx\subparagraph\undefined\else
\let\oldsubparagraph\subparagraph
\renewcommand{\subparagraph}[1]{\oldsubparagraph{#1}\mbox{}}
\fi

%%% Use protect on footnotes to avoid problems with footnotes in titles
\let\rmarkdownfootnote\footnote%
\def\footnote{\protect\rmarkdownfootnote}

%%% Change title format to be more compact
\usepackage{titling}

% Create subtitle command for use in maketitle
\newcommand{\subtitle}[1]{
  \posttitle{
    \begin{center}\large#1\end{center}
    }
}

\setlength{\droptitle}{-2em}

  \title{Packages \& Reproducibility: Install what you need, attach what you want}
    \pretitle{\vspace{\droptitle}\centering\huge}
  \posttitle{\par}
    \author{Mike DeCrescenzo}
    \preauthor{\centering\large\emph}
  \postauthor{\par}
      \predate{\centering\large\emph}
  \postdate{\par}
    \date{2018-05-30}


\usepackage{amsthm}
\newtheorem{theorem}{Theorem}[section]
\newtheorem{lemma}{Lemma}[section]
\theoremstyle{definition}
\newtheorem{definition}{Definition}[section]
\newtheorem{corollary}{Corollary}[section]
\newtheorem{proposition}{Proposition}[section]
\theoremstyle{definition}
\newtheorem{example}{Example}[section]
\theoremstyle{definition}
\newtheorem{exercise}{Exercise}[section]
\theoremstyle{remark}
\newtheorem*{remark}{Remark}
\newtheorem*{solution}{Solution}
\begin{document}
\maketitle

{
\setcounter{tocdepth}{2}
\tableofcontents
}
\emph{(Note: An earlier version of this post referred to ``loading''
packages when I really meant ``attaching.'' Thanks to
\href{https://twitter.com/thosjleeper/status/1001859564113924096}{Thomas
Leeper} for the clarification.)}

\begin{center}\rule{0.5\linewidth}{\linethickness}\end{center}

When we distribute R code (for publication/replication archives, on
Github, through blog posts, etc), we like the code to run smoothly on
someone else's machine. Packages present a nominal problem because
different users have different packages installed on their computer.
Ideally the script we are distributing should install dependencies
without the redundancy of re-installing packages that a user already has
installed.

To solve this, many users conditionally install packages using
\texttt{require()}; if a package fails to attach, it is installed.
Supposing that we want to use a package called \texttt{pkg}\ldots{}

\begin{Shaded}
\begin{Highlighting}[]
\CommentTok{# if require() succeeds, package is attached}
\CommentTok{# if require() fails, package is installed}
\ControlFlowTok{if}\NormalTok{ (}\KeywordTok{require}\NormalTok{(}\StringTok{"pkg"}\NormalTok{) }\OperatorTok{==}\StringTok{ }\OtherTok{FALSE}\NormalTok{) \{}
  \KeywordTok{install.packages}\NormalTok{(}\StringTok{"pkg"}\NormalTok{)}
\NormalTok{\}}
\end{Highlighting}
\end{Shaded}

We see often see this in (for example) package ReadMe files on Github.
But for distributing bigger projects, the approach has two shortcomings.

\begin{enumerate}
\def\labelenumi{\arabic{enumi}.}
\tightlist
\item
  Sometimes we require a package, but we don't want to attach it fully.
  We may only need it for a function or two, or we want to prevent
  \href{https://github.com/r-lib/conflicted}{clashing function names} in
  projects that use many packages, and so we prefer to use the
  \texttt{pkg::function()} grammar instead of attaching every package
  with \texttt{library()}.
\item
  If we require several packages for a large project (like a replication
  archive for a published journal article), including code to install
  all packages ends up unsightly and repetitive (but functional, to be
  fair).
\end{enumerate}

This post discusses a straightforward strategy for addressing this.

\hypertarget{check-for-installed-packages-without-attaching-them}{%
\subsection{Check for installed packages without attaching
them}\label{check-for-installed-packages-without-attaching-them}}

Rather than use \texttt{require()}, which has the side-effect of
attaching an installed package, we can look for installed packages using
the output from \texttt{installed.packages()}. In particular, the
function should return an array whose rownames contain the names of our
installed packages.

\begin{Shaded}
\begin{Highlighting}[]
\KeywordTok{head}\NormalTok{(}\KeywordTok{rownames}\NormalTok{(}\KeywordTok{installed.packages}\NormalTok{()))}
\CommentTok{## [1] "AER"         "Amelia"      "BH"          "Brobdingnag" "DBI"        }
\CommentTok{## [6] "DEoptimR"}
\end{Highlighting}
\end{Shaded}

Using these rownames, we can check for installed packages without
attaching those packages incidentally. Just create a character vector of
required packages and cross-reference those names with the list of
installed packages.

\begin{Shaded}
\begin{Highlighting}[]
\KeywordTok{c}\NormalTok{(}\StringTok{"ggplot2"}\NormalTok{, }\StringTok{"beepr"}\NormalTok{, }\StringTok{"some_other_package"}\NormalTok{) }\OperatorTok\StringTok{ }\KeywordTok{rownames}\NormalTok{(}\KeywordTok{installed.packages}\NormalTok{())}
\CommentTok{## [1]  TRUE  TRUE FALSE}
\end{Highlighting}
\end{Shaded}

\hypertarget{install-what-we-need-attach-what-we-want}{%
\subsection{Install what we need, attach what we
want}\label{install-what-we-need-attach-what-we-want}}

The structure laid out above can be augmented to install missing
packages. Just use a little bit of indexing:

\begin{Shaded}
\begin{Highlighting}[]
\CommentTok{# vector of requirements }
\NormalTok{requires <-}\StringTok{ }\KeywordTok{c}\NormalTok{(}\StringTok{"pkg_1"}\NormalTok{, }\StringTok{"pkg_2"}\NormalTok{)}

\CommentTok{# evaluates to TRUE if not already installed}
\NormalTok{to_install <-}\StringTok{ }\NormalTok{(requires }\OperatorTok\StringTok{ }\KeywordTok{rownames}\NormalTok{(}\KeywordTok{installed.packages}\NormalTok{()) }\OperatorTok{==}\StringTok{ }\OtherTok{FALSE}\NormalTok{)}

\CommentTok{# install missing packages}
\NormalTok{cloud_url <-}\StringTok{ "https://cloud.r-project.org/"}
\KeywordTok{install.packages}\NormalTok{(requires[to_install], }\DataTypeTok{repos =}\NormalTok{ cloud_url)}
\end{Highlighting}
\end{Shaded}

Because we never call \texttt{require()}, we can attach (or not attach)
whichever packages we want with subsequent \texttt{library()} commands.
(We could have also used \texttt{requireNamespace()} to check for
packages.)

\hypertarget{coda-on-the-use-of-library-vs-require}{%
\section{\texorpdfstring{Coda: on the use of \texttt{library()} vs
\texttt{require()}}{Coda: on the use of library() vs require()}}\label{coda-on-the-use-of-library-vs-require}}

Thinking back to this
\href{https://yihui.name/en/2014/07/library-vs-require/}{oft-cited post
by Yihui Xie}, the takeaway was that we should only be using
\texttt{require()} to\ldots{}

\begin{enumerate}
\def\labelenumi{\arabic{enumi}.}
\tightlist
\item
  conditionally install packages, or
\item
  implement ``bonus-features'' that may enhance a package or function
  but that aren't strictly necessary
\end{enumerate}

In reference to {[}1{]}, I think we can (and should) flatly avoid
\texttt{require()} in any situation where we need a package but don't
want to attach it. For me, this is basically all of my research
projects, and I think many other users will find themselves in a similar
boat. As long as \texttt{require} attaches a package in the process of
checking for it, \texttt{require} gives unintended and unnecessary
side-effects.

In reference to {[}2{]}, it's a little funny that we got to a point
where we would only use a function called \texttt{require} for features
that are strictly not required. It's quite the contradiction. In a
parallel universe we might have named the function something squishier
like \texttt{suppose()}. At any rate, the chief reason to use
\texttt{require} in this case---it returns logical output---isn't
necessary either, since \texttt{library()} has a \texttt{logical.return}
argument that achieves the same objective.

\hypertarget{thanks-for-reading}{%
\section{Thanks for reading}\label{thanks-for-reading}}

Feel free to \href{https://www.twitter.com/mikedecr}{get in touch} with
comments, links to related posts by others, packages that implement
similar features, and so on.

I just finished a major website overhaul. While it took only about one
day of work, I've had my eyes open for a new
\href{https://themes.gohugo.io/}{theme} for
\href{https://gohugo.io/}{Hugo} for a while.

I mean, a \emph{while}. I would download a theme, give it a 15-minute
test run offline, and decide that it wasn't for me. I did that maybe a
dozen times.\footnote{The only way I was able to test themes so
  efficiently was thanks to the amazing
  \href{https://bookdown.org/yihui/blogdown/}{blogdown} package for R.}

But this is mainly my fault. I am a bad combination of \emph{picky} and
\emph{unskilled} with websites and web design, which made me a fickle
chooser throughout the process. So to make up for that fickleness, I
wanted to acknowledge and promote some of the Hugo themes that I really
liked, but ultimately did not choose for my site.

\hypertarget{goa}{%
\subsection{Goa}\label{goa}}

I previously used \href{https://themes.gohugo.io/hugo-goa}{Goa} for my
site. It is simple, sensible, and easy to modify.

\hypertarget{introduction}{%
\subsection{Introduction}\label{introduction}}

One of two themes by \href{https://vickylai.com/}{Vicky Lai} that I will
promote.
\href{https://themes.gohugo.io/hugo-theme-introduction/}{Indroduction}
has a bold design and a layout that is focused on the landing page,
which I enjoyed. I especially loved the ``About'' section.

\hypertarget{call-me-sam}{%
\subsection{Call Me Sam}\label{call-me-sam}}

The other Vicky Lai theme. If ``Introduction'' was bold, I would say
that \href{https://themes.gohugo.io/hugo-theme-sam/}{Call Me Sam} is
\emph{brave}. I don't trust that my description will do it justice;
better to \href{http://vickylai.com/call-me-sam}{see for yourself}.

\hypertarget{cocoa-enhanced}{%
\subsection{Cocoa Enhanced}\label{cocoa-enhanced}}

\href{https://themes.gohugo.io/cocoa-eh-hugo-theme/}{Cocoa Enhanced} is
crisp and clean. Its type design is opinionated but flexible. I was most
tempted to use this theme.

\hypertarget{coder}{%
\subsection{Coder}\label{coder}}

I must include \href{https://themes.gohugo.io/hugo-coder/}{Coder}
because it is the inspiration for the theme I ultimately chose. Why do I
like it? The simple landing page. The navigation bar is out of the way
but present. The backend tension between ``section'' pages and ``post''
archives (one of the more difficult aspects for a newbie working with
Hugo) is easily to navigate.

\hypertarget{let-me-know-about-your-theme-that-you-really-like}{%
\section{Let me know about your theme that you really
like}\label{let-me-know-about-your-theme-that-you-really-like}}

Perhaps on \href{https://twitter.com/mikedecr}{Twitter}

\begin{Shaded}
\begin{Highlighting}[]
\KeywordTok{print}\NormalTok{(}\StringTok{"ello"}\NormalTok{)}
\end{Highlighting}
\end{Shaded}

Helpful animated \#dataviz showing what happens to the slope of one
coefficient in a model when controlling for other variables in multiple
regression(\#rstats code: https://t.co/yhVLj325Oh)
pic.twitter.com/2foYfXDo28

--- 🎃 Andrew Heiss, scary PhD 🦇 (@andrewheiss) October 18, 2018

Ok fixed. In this fig, y is (beta * humidity), plus the regression
residual. This is equivalent to starting with the fully estimated
regression and subtracting out terms for every other covariate
pic.twitter.com/fLs4WxHTaK

--- Michael DeCrescenzo (@mikedecr) October 18, 2018

omg this is astounding! Do you have the code for it?

--- 🎃 Andrew Heiss, scary PhD 🦇 (@andrewheiss) October 18, 2018


\end{document}
